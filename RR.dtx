% \iffalse meta-comment
% 
% RR.dtx Copyright (C) 1997-2002 J. Grimm INRIA/MIAOU
% Copyright (C) 2004 2006 2007 2010 J. Grimm INRIA/APICS
% Copyright (C) 2011 J. Grimm INRIA/MARELLE
% (c) Philippe Louarn - INRIA/IRISA - 1993-1997
% Fichier PS de couverture du a G. Ouanounou - INRIA Rocq.
% (c) Inria pour les logos
% License change to LPPL (may 2007)
% $Id: RR.dtx,v 1.23 2012/01/17 17:11:12 grimm Exp grimm $
% 
%\fi
% 
% \CheckSum{1313}
% \CharacterTable
%  {Upper-case    \A\B\C\D\E\F\G\H\I\J\K\L\M\N\O\P\Q\R\S\T\U\V\W\X\Y\Z
%   Lower-case    \a\b\c\d\e\f\g\h\i\j\k\l\m\n\o\p\q\r\s\t\u\v\w\x\y\z
%   Digits        \0\1\2\3\4\5\6\7\8\9
%   Exclamation   \!     Double quote  \"     Hash (number) \#
%   Dollar        \$     Percent       \%     Ampersand     \&
%   Acute accent  \'     Left paren    \(     Right paren   \)
%   Asterisk      \*     Plus          \+     Comma         \,
%   Minus         \-     Point         \.     Solidus       \/
%   Colon         \:     Semicolon     \;     Less than     \<
%   Equals        \=     Greater than  \>     Question mark \?
%   Commercial at \@     Left bracket  \[     Backslash     \\
%   Right bracket \]     Circumflex    \^     Underscore    \_
%   Grave accent  \`     Left brace    \{     Vertical bar  \|
%   Right brace   \}     Tilde         \~}
%
% \iffalse
% \section{Identification}
% Test de la version de latex.
%    \begin{macrocode}
%<*RR>
\def\RRfiledate{2012/02/16}
\def\RRfileversion{5.1c}
\NeedsTeXFormat{LaTeX2e}
%</RR>
%<RR>\ProvidesPackage{RR}
%<RR>       [\RRfiledate\space v\RRfileversion\space
%<RR>         Copyright INRIA/MIAOU/APICS/MARELLE 2002-2011]
%<*driver>
\ProvidesFile{RR.drv}[\RRfiledate\space v\RRfileversion\space]
%</driver>
%<*driver>
\documentclass[a4paper,twoside]{ltxdoc}
\makeatletter
\def\glossary@prologue{\section{Historique des changements}}
\def\index@prologue{\section{Index}%
  Tous les nombres correspondent \`a des num\'eros de ligne dans le code source.}
\makeatother 
\usepackage{RR}
\IfFileExists{bera.sty}{\usepackage{bera}}{}
%\usepackage[urw-garamond]{mathdesign}
\usepackage{hyperref}
\usepackage[frenchb]{babel}
%    \end{macrocode}
%
%    On supprime un certain nombre de mots dans l'index.
%    \begin{macrocode}
\DoNotIndex{\@minus,\@plus,\\,\[,\\,\],\@listi,\@setfontsize}
\DoNotIndex{\abovedisplayshortskip,\abovedisplayskip,\addtolength,\advance}
\DoNotIndex{\belowdisplayshortskip,\belowdisplayskip}
\DoNotIndex{\begin,\begingroup,\bf,\bfdefaults,\bfseries,\bigskip}
\DoNotIndex{\def,\DeclareFontFamily,\DeclareFontShape}
\DoNotIndex{\DeclareMathAlphabet,\DeclareMathSizes,\DeclareOldFontCommand}
\DoNotIndex{\DeclareOption,\DeclareSymbolFont}
\DoNotIndex{\edef,\em,\end,\endgroup,\endlist}
\DoNotIndex{\fill,\frac,\gdef,\global,\',\^,\`}
\DoNotIndex{\ifdim,\ifcase,\or,\ifnum,\ifodd,\else,\ifx,\fi,\fi,\fi,\fi,\fi}
\DoNotIndex{\kern,\let,\newcommand,\newdimen,\p@}
\DoNotIndex{\relax,\renewcommand,\setlength,\z@}
%    \end{macrocode}
%    On veut un index, avec des nombres de lignes
\let\disablecrossrefs\DisableCrossrefs
\EnableCrossrefs
\CodelineIndex

%    \begin{macrocode}
\GetFileInfo{RR.drv}
%\DisableCrossrefs % Say \DisableCrossrefs if index is ready
\RecordChanges     % Gather update information
%\OnlyDescription  % uncomment for user instructions only

\hfuzz 1pt
\textheight22cm
\footskip1cm
\begin{document}
 \hfuzz5pt
 \DocInput{RR.dtx}
 \hfuzz15pt \hbadness=7000
\PrintIndex
\PrintChanges
\end{document}
%</driver>
%    \end{macrocode}
% \fi
% \selectlanguage{french}
% \changes{v3.4}{1997/09/17}{Version initiale}
% \changes{v4.3}{2006/05/03}{conversion en 7bits}
% \changes{v5.0}{2011/09/28}{Name chamges @RR gives RR@ }
% \RRtitle{Les styles \texttt{RR} et \texttt{RRA4}\thanks{Ce package
%         a pour num\'ero de version \RRfileversion, a \'et\'e
%        modifi\'e le \RRfiledate.}}
% \RRetitle{The syle files \texttt{RR} and \texttt{RRA4}}
% \RRequipe{MARELLE}
% \RCSophia
% \RRauthor{Jos\'e Grimm}
% \RRresume{
% Ce document d\'ecrit le fichier de style |RR.sty|, 
% qui permet de saisir les
% rapports de recherche et les rapports techniques Inria. Il mentionne
% \'egalement les styles |RRA4| et |RRthemes| qui ont exist\'e dans le pass\'e.
% Il pr\'esente les recommandations de saisie, comment obtenir un num\'ero de
% rapport ainsi que la fa\c con de d\'eposer la version finale
% (PostScript ou Pdf)
% sur le serveur HAL Inria.
% }
%
% \RRmotcle{
% rapport de recherche, rapport technique,
% documentation scientifique, Inria, LaTeX}
%
% \RRabstract{
% This documents describes the source of the style |RR.sty|, and how to
% use it for typesetting Inria research reports or technical
% reports by using LaTeX, the way to obtain a serial number
% and how to deposit the final version (PostScript or Pdf file) on
% HAL. We also mention the obsolete |RRA4| and |RRthemes| packages.
% }
%
% \RRkeyword{
% research report, technical report, Inria, LaTeX}
%
% \RRnote{Premier auteur : Philippe Louarn}
% \RRnote{
% Le style \LaTeX{} d\'ecrit dans ce document est distribu\'e tel quel.
% Il a \'et\'e test\'e avec la version de \LaTeX\ en service
% \`a Sophia le \today, mais il peut n\'ecessiter quelques adaptations
% selon les configurations des autres sites Inria.
% }
% \RRnote{Le fichier de style utilise un encodage 7 bits}
% \RRdate{November 1997}
% \RRversion{2}
% \RRdater{\today}
% \RRNo{7002}\makeRT
% \addtolength\topmargin{-1.5cm}
% \addtolength\textheight{1cm}
% \setlength\oddsidemargin{2.5cm}
% \setlength\evensidemargin{2.5cm}
% \begingroup\catcode`\|=11\gdef\Vbar{|}\endgroup
% \newenvironment{decl}%
%    {\par\small\addvspace{4.5ex plus 1ex}%
%     \vskip -\parskip
%     \noindent\hspace{-\leftmargini}%
%     \begin{tabular}{\Vbar l\Vbar}\hline\ignorespaces}%
%    {\\\hline\end{tabular}\par\nobreak
%     \vspace{2.3ex}\vskip -\parskip}
% \newcommand{\m}[1]{\mbox{$\langle$\it #1\/$\rangle$}}
% \renewcommand{\arg}[1]{{\tt\string{}\m{#1}{\tt\string}}}
% \renewcommand{\oarg}[1]{{\tt[}\m{#1}{\tt]}}
% \tableofcontents
% \section*{Pr\'eface}
% 
% La r\'eflexion sur la communication scientifique
% de l'Inria a entra\^\i n\'e une s\'erie d'actions, certaines spectaculaires
% (le 25\textsuperscript{e} anniversaire, le changement de logo, la cr\'eation de l'Ucis,
% etc.), d'autres
% moins. La politique \'editoriale de l'institut
% vis \`a vis du monde ext\'erieur (communaut\'e scientifique et monde industriel)
% se doit d'\'evoluer. De plus, la mise \`a dispostion des services de documentation
% d'outils modernes (ftp, wais, etc.) montre que les documents scientifiques
% ne peuvent plus se suffire du seul support papier.
% Il est apparu n\'ecessaire d'harmoniser l'ensemble des
% publications de l'institut. 
%
% Nous avons d\'efini un style \LaTeX{} pour la saisie des rapports
% de recherche Inria qui respecte un certain nombre de crit\`eres :
% \begin{itemize}
% \item respect de l'identit\'e visuelle de l'institut telle que d\'efinie
% dans la charte graphique de l'Inria (logos, couleurs), 
% en particulier pour les versions des documents 
% qui seront disponibles sur les serveurs
% de fichiers ;
% \item respect autant que possible des normes de l'\'edition scientifique ;
% \item rester le plus proche possible des habitudes des auteurs ;
% \item harmoniser les divers rapports ;
% \item pour l'unit\'e de recherche de Rennes, 
% faciliter le passage d'une publication interne Irisa en rapport
% de recherche Inria.
% \end{itemize}
%
% Vous trouverez des informations techniques concernant la r\'ealisation d'un 
% rapport sur le serveur interne de l'Inria \`a l'URL suivante :\\
% \href{http://www.inria.fr/interne/disc/publier/rrrt/index.html}{http://www.inria.fr/interne/disc/publier/rrrt/index.html}
% \section*{M\'eta-donn\'ees sur le serveur Web Inria}
% Les fiches de m\'eta-donn\'ees sont depuis mai 2004 converties en XML par le logiciel
% \textit{Tralics}. Ce logiciel comprend les commandes \LaTeX\ usuelles,
% mais les interpr\`ete parfois de fa\c con \'etrange. Il est d\'econseill\'e
%  de mettre trop de formules math\'ematiques dans le r\'esum\'e. Il vaut 
% mieux ne pas mettre de r\'ef\'erences bibliographiques. 
%
% Dans la version actuelle de \textit{Tralics}, les r\'esum\'es, titres, etc.,
% sont traduits imm\'ediatement (contrairement \`a \LaTeX, qui les met de 
% c\^ot\'e et ne les traduit que lors de l'apparition de la command |\makeRR|
% ou |\makeRT|). Il est donc imp\'eratif de d\'efinir toutes les commandes 
% utilis\'ees dans les r\'esum\'es, titre, mots-cl\'es etc, avant leur
% utilisation. 

% \section*{Pr\'eface de la version 4.10}
% \`A partir du 22 mai 2006, le mode de publication d'un rapport INRIA
% \'evolue. Tout nouveau rapport INRIA sera dor\'enavant d\'epos\'e dans 
% l'Archive ouverte de l'Institut, tandis que l'ensemble de leur collection y
% sera transf\'er\'e. Le rapport INRIA n\'ecessitera cependant toujours la 
% validation scientifique du chef d'\'equipe, avec le m\^eme niveau 
% d'exigence \'editoriale et scientifique. Il continuera \'egalement \`a 
% enrichir la collection de l'Institut.  Les m\'eta-donn\'ees ne sont plus 
% produites par \textit{Tralics}, mais par l'auteur du document.
%
% \`A l'occasion des quarante ans de l'Inria, certains changements sont
% apparus. Les projets sont devenus des \'equipes projets, les unit\'es de
% recherche sont redevenus des centres de recherche, et Futurs a \'et\'e
% remplac\'e par trois centres. 
% \section*{Pr\'eface de la version 5.0}
% En 2011, Inria s'est dot\'e d'une nouvelle identit\'e visuelle et la
% couleur de la page de titre des rapports de recherche est redevenue rouge.
% La page de garde contient le nom de l'EPI au lieu de son th\`eme de recherche.
% Les fichiers de style |RRthemes| et |RRA4| ont disparu.
% Le package se pr\'esente sous forme d'une archive tar, qui contient les
% fichiers images PostScript et Pdf, le fichier de style, un fichier exemple,
% la documentation et les deux fichiers |RR.ins| et |RR.dtx| qui ont servi \`a
% cr\'eer le style et la documentation. Pour se servir du style il suffit de
% mettre le contenu de l'archive dans un r\'epertoire qui fait partie du
% chemin de recherche  de \LaTeX. Pour d\'eposer un rapport en version source
% sur HAL, il faut \'egalement d\'eposer le fichier de style et les
% images utilis\'ees par le document (le plus simple est toutes les mettre).
% Le style ne red\'efinit plus des valeurs comme |\vfuzz|  et |\topfraction|.
% Le bas des page contient maintent |Inria|.
% \section*{Notes}
%
% Ce document utilise la commande |\usepackage{bera}|, qui charge 
% la fonte \mbox{BitStream} Vera. 
%
%
% \StopEventually{
% \begin{thebibliography}{9}
%
% \bibitem{companion} Michel Goossens, Frank Mittelbach and Alexander 
% Samarin, \textit{The \LaTeX{} Companion}, Addison-Wesley, Reading,
% Massachusetts, 1994.
%
% \bibitem{texbook} Donald Knuth, \textit{The \TeX book}, Addison-Wesley,
% Reading, Massachusetts, 1983, revised in 1993.
%
% \bibitem{latexbook} Leslie Lamport, \textit{\LaTeX: A Document
% Preparation System}, 2nd ed., Addison-Wesley, Reading, Massachusetts,
% 1994. 
%
% \bibitem{charte} UCIS, \textit{Charte graphique} et \textit{Abr\'eg\'e de la
% charte graphique}, INRIA, 1994. 
% \end{thebibliography}
% } 
% \section{Description du style}
% \begin{decl}|\usepackage[french,T1,OT1,noinputenc,utf8]{RR}|\end{decl}
% Historiquement, le paquet fournissait deux styles |RR| et |RRA4|, ce dernier
% ayant une zone de texte nettement plus grande. Depuis 2007, le style ne
% red\'efinit plus la taille de texte (les anciennes tailles sont en
% commentaire \`a la fin), de telle sorte que les  deux styles |RR| et
% |RRA4| sont devenus interchangeables, et |RRA4| a  \'et\'e supprim\'e en
% 2011.  Le style |RRthemes| a \'et\'e  cr\'e\'e en 2010, en tant que
% compl\'ement optionnel, \`a charger apr\`es |RR|, car ce style red\'efinit
% certaines commandes. Il a \'et\'e supprim\'e en 2011 et il ne faut plus
% l'utiliser. Toutes les options, sauf |french| sont obsol\`etes.
%
% Toutes les commandes fournies par le style sont expliqu\'ees dans ce
% document. On donne aussi le code source d'un fichier d'exemple. 
%
% \textbf{Options historiques} : Les deux options |T1| ou |OT1|
% \'etaient des param\`etres du package |fontenc|, qui n'est plus charg\'e
% automatiquement. Les options |utf8| et |noinputenc| contr\^olaient
% l'encodage d'entr\'ee (param\`etres de |inputenc|). 
% 
% \textbf{Options utilisables} :
% La seule option utilisable est donc l'option |french|, \`a utiliser  si votre
% document est en fran\c cais, mais que le style |RR| pense que 
% c'est de l'anglais.
%
%    \begin{macrocode}
%<*sample>
\documentclass[twoside]{article}
\usepackage[a4paper]{geometry}
\usepackage[latin1]{inputenc} % ou \usepackage[utf8]{inputenc}
\usepackage[T1]{fontenc} % ou \usepackage[OT1]{fontenc}
\usepackage{RR}
%    \end{macrocode}
% L'utilisateur peut bien entendu charger d'autres styles, ou passer des
% options \`a la commande |\documentclass|. Noter que |a4paper| est une option
% qui explique \`a |pdflatex| d'utiliser du papier format A4, et non du papier
% format Letter am\'ericain. L'option |twoside| demande \`a \LaTeX\ de mettre
% des en-t\^etes diff\'erents sur les pages paires et impaires (auteurs sur
% les pages paires, titre sur les pages impaires). Dans notre exemple, nous
% allons charger le package |hyperref|.
%    \begin{macrocode}
\usepackage{hyperref}
%    \end{macrocode}
% \begin{decl} |\usepackage{french}|\\
%  |\usepackage[frenchb]{babel}|\\
%  |\usepackage[french]{babel}| \end{decl}
% L'utilisation de l'une de ces commandes a pour but de d\'eclarer que la langue
% principale du document est le fran\c cais. Dans le cas contraire, il s'agit de
% l'anglais. La page de titre est dans la langue principale. Dans
% l'exemple qui suit, la, langue principale et la page de titre sont en
% anglais.  
%    \begin{macrocode}
%%\usepackage[frenchb]{babel} % optionnel
%    \end{macrocode}
% \begin{decl} |\RRNo|\arg{num} \end{decl}
% La commande |\RRNo| positionne le num\'ero du rapport. Depuis le 26
% juin 2008,  ce num\'ero est attribu\'e via le formulaire
% |https://rrrt.inria.fr/AttribNum| Avant cette date, le
% SICS\footnote{devenu UCIS puis DISC} ins\'erait un num\'ero unique
% dans le PostScript. 
% \changes{v5.0}{2011/09/28}{rrno added}
%    \begin{macrocode}
\RRNo{7003}
%    \end{macrocode}
% \begin{decl} |\RRnbpage| \arg{num} \end{decl}
% Normalement, \LaTeX\ calcule correctement le nombre de pages. On n'utilisera
% donc cette macro qu'en dernier recours.
% \begin{decl} |\RRdate| \arg{date} \end{decl}
% Cette commande est optionnelle, mais si vous ne l'utilisez pas, vous aurez un
% message d'avertissement de \LaTeX. L'argument \m{date} est en principe
% form\'e du nom du mois et de l'ann\'ee de la publication.
% \changes{v4.2}{2004/11/15}{Comments}
% \changes{v5.0}{2011/09/28}{Dates in English}
%    \begin{macrocode}
%%
%% date de publication du rapport
\RRdate{September 1997} 
%    \end{macrocode}
% \changes{v4.5}{2006/11/14}{Description added}
% \begin{decl} |\RRversion| \arg{N} \end{decl}
% Cette commande est \`a utiliser dans le cas o\`u le document courant n'est 
% pas la version originale mais la version N (avec N au moins 2).
% \begin{decl} |\RRdater| \arg{date} \end{decl}
% Ceci indique la date de r\'evision, dans le cas o\`u il s'agit d'une 
% version deux ou plus grande. 
%    \begin{macrocode}
%%
%% Cas d'une version deux
%% \RRversion{2} 
%% date de publication de la version 2
%% \RRdater{November 2008} 
%    \end{macrocode}
% \begin{decl} |\RRauthor| \arg{auteur} \end{decl}
% Cette commande donne le nom de l'auteur du rapport. Bien entendu, l'argument
% est form\'e du pr\'enom et du nom (complets). On peut
% \'eventuellement mettre des notes, au moyen de la commande
% |\thanks|. S'il y a plusieurs auteurs, on les 
% s\'epare par la commande |\and|.
% \changes{v4.2}{2004/11/15}{Comments}
%    \begin{macrocode}
%%
\RRauthor{% les auteurs
 % Premier auteur, avec une note
Jos\'e Grimm\thanks{Footnote for first author}%
  % note partag\'ee (optionnelle)
  \thanks[sfn]{Shared foot note}%
 % \and entre chaque auteur s'il y en a plusieurs
  \and 
Philippe Louarn\thanks{Footnote for second author}%
 % r\'ef\'erence \`a la note partag\'ee 
\thanksref{sfn}
 % liste longue pour tests de mise en page
\and Nicolas Bourbaki\thanks{etc} \and Andr\'e Lichnerowicz 
\and Marcel-Paul Sch\"utzenberger \and Jacques-Louis Lions}
%    \end{macrocode}
% \begin{decl} |\thanks| \arg{note} \\
% |\thanks| \oarg{cle}\arg{note}\\
% |\thanksref|\arg{cle}
% \end{decl}
% La commande |\thanks| permet d'accrocher une note en bas de page \`a un auteur.
% Si vous voulez partager la m\^eme note entre deux auteurs, utilisez pour le 
% premier la commande |\thanks| avec un argument optionnel, et pour le second,
% la commande |\thanksref|.
%
% \begin{decl} |\authorhead| \arg{auteur-court} \end{decl}
% L'en-t\^ete des pages paires contient la liste des auteurs. Dans le cas qui
% nous int\'eresse, cette liste est tellement longue
% que cela ne tient pas sur la page. On peut donc utiliser la macro
% |\authorhead| pour placer \m{auteur-court} sur les pages paires.
% \changes{v4.2}{2004/11/15}{Comments}
%    \begin{macrocode}
%% Ceci apparait sur chaque page paire.
\authorhead{Grimm \& Louarn \& others} 
%    \end{macrocode}
% \begin{decl} |\RRtitle| \arg{titre-f} \end{decl}
% La commande |\RRtitle| d\'efinit le titre fran\c cais du rapport.
% \changes{v4.2}{2004/11/15}{Comments}
%    \begin{macrocode}
%% titre francais long 
\RRtitle{Exemple de document\\ 
 utilisant le style\\rapport de recherche\\Inria}
%    \end{macrocode}
% \begin{decl} |\RRetitle| \arg{titre-e} \end{decl}
% Cette macro d\'efinit le titre anglais. Rappelons que le titre du document est
% le titre dans la langue principale. Si on est en fran\c cais, la page de
% couverture, de m\^eme que la page~1 contiendra le titre fran\c cais, et la page~2
% contiendra le titre anglais. Si on est en anglais, la page de
% couverture, de m\^eme que la page~1 contiendra le titre anglais, et la page~2
% contiendra le titre fran\c cais.
% \changes{v4.2}{2004/11/15}{Comments}
%    \begin{macrocode}
%% English title
\RRetitle{Very Long Title for Checking Whether it Fits on the Cover 
  Page of Inria's Research Report Document Type (Extended Version)}
%    \end{macrocode}
% \begin{decl} |\titlehead| \arg{titre-court} \end{decl}
%  Si le document est en fran\c cais, \m{titre-f} sera sur toutes
% les pages impaires du rapport, si le document est en anglais, ce sera
% \m{titre-e}. Dans le cas o\`u le titre est tr\`es long, on peut utiliser
% |\titlehead|, de sorte que ce sera la valeur de \m{titre-court} qui sera
% utilis\'ee. Si on veut utiliser cette commande, il faut le faire apr\`es
% |\RRtitle|, |\RRetitle|, et bien entendu avant |\begin{document}|.
% \changes{v4.2}{2004/11/15}{Comments}
%    \begin{macrocode}
%%
% %titre court, sur toutes les pages.
\titlehead{Example of RR.sty}
%    \end{macrocode}
% \begin{decl} |\RRnote| \arg{note} \end{decl}
% Cette commande permet de mettre des notes sur la premi\`ere page (page de
% titre int\'erieure). Il est
% d\'econseill\'e d'utiliser la commande |\footnote| sur la premi\`ere page. 
%    \begin{macrocode}
%%
\RRnote{This is a note}
\RRnote{This is a second note}
%    \end{macrocode}
% \begin{decl} |\RRresume| \arg{resume-f} \\
% |\RRabstract| \arg{resume-e} \end{decl}
% Ces deux commandes d\'efinissent le r\'esum\'e fran\c cais et anglais. 
% Le r\'esum\'e de
% la langue principale est sur la page~1, l'autre sur la page~2. On peut aussi
% utiliser les commandes |\resume| et |\abstract|, qui sont de simples alias.
%
% Les r\`egles Afnor fixent la taille maximum d'un r\'esum\'e \`a 250 mots, en 
% un seul paragraphe. Il faut aussi \'eviter les environnement itemize, etc,
% les r\'ef\'erences bibliographiques, et les formules math\'ematiques.
% Il ne faut pas oublier que le r\'esum\'e est diffus\'e \emph{tel quel}, sur le
% serveur WEB\footnote{Apr\`es traduction en XML par Tralics}%
% \footnote{Dans le cas d'un d\'epot sur HAL, l'auteur fournit, en plein
% texte, une version unique de son r\'esum\'e}
% et le bulletin d'annonces des rapports papier. S'il est parfois
% impossible d'\'eviter des phrases telles que \og on montre que la fonction
% d\'efinie par Dupond et Dupont est dans |$|H|^|2|$| \fg, il est
% inadmissible de voir \og on montre ici que la relation 
% |\|label|{|eq-joli|}| propos\'ee par |\|cite|{|DdDt98|}| est \'egalement vraie si
% ...\fg 
% \changes{v4.2}{2004/11/15}{Warning added}
%    \begin{macrocode}
%%
\RRresume{Ce document montre comment utiliser le style RR.sty.
Pour en savoir plus, consulter le fichier RR.dvi ou RR.pdf.
D\'efinissez toujours les commandes avant utilisation.
}
\RRabstract{
Our society relies increasingly on digital technologies to communicate,
seek medical information, travel, or have fun. These often-invisible
technologies simplify our tasks and enrich our daily lives, while also 
developing the economy. At the interface of computer science and 
mathematics, from pure research to technological development and to 
industrial transfer, researchers at Inria, a public research institute, 
are inventing tomorrow's digital technologies. Inria's research is 
collaborative, which is evidenced by the diversity of the talent 
comprising its research teams, as well as in the many joint projects 
conducted with public and private research entities in France and 
abroad. While competing with the leading international specialists in 
their field, Inria researchers and staff are also committed to sharing 
their knowledge with the widest possible audience.

Computational technologies are currently used to increase the 
efficiency and safety of our transport services, introduce smart 
applications into homes, and develop agricultural practices that are 
better for the environment... They are also behind a range of new 
services, and are bringing about fundamental changes in the ways we 
live our lives, improving them on a day-to-day basis. }
%    \end{macrocode}
% \begin{decl} |\RRmotcle| \arg{mot-cle-f} \\
%  |\RRkeyword| \arg{mot-cle-e} \end{decl}
% Ces deux commandes permettent de d\'efinir les mots cl\'es dans les deux
% langues. 
%    \begin{macrocode}
%%
\RRmotcle{calcul formel, base de formules,  %mots-cl\'es sans rapport
protocole, diff\'erentiation automatique, % les uns avec les autres
g\'en\'eration de code,  mod\'elisation, lien symbolique/num\'erique,
matrice structur\'ee, r\'esolution de syst\`emes polynomiaux}
\RRkeyword{task placement, environment, % keywords (taken from some
 message passing,  network management,   % random research report)
 performances} 
%    \end{macrocode}
% \begin{decl} |\RRprojet| \arg{projet} \\
% |\RRprojets| \arg{projets} \\
% |\RRequipe| \arg{projet} \end{decl}
% La commande |\RRprojet| permet de d\'efinir le projet de l'auteur. 
% Dans le cas  o\`u il y a plusieurs auteurs, de plusieurs projets 
% diff\'erents, on utilisera la seconde commande |\RRprojets|,
% on mettra \og et \fg\ entre  les deux projets. Depuis 2007, la
% d\'enomination a chang\'e, les \'equipes-projets peuvent utiliser les
% commande |\RRequipe| ou |\RRprojet|.
%    \begin{macrocode}
%% 
%% \RRprojet{Apics}  % cas d'un seul projet
\RRprojets{Apics and Op\'era and Marelle} 
%    \end{macrocode}
% \begin{decl} |\RRtheme| \arg{theme} \end{decl}
% L'argument est un ou plusieurs
% parmi |\THCom|, |\THCog|, |\THSym|,  |\THNum| et  |\THBio|.
% Ces th\`emes sont \`a utiliser \`a partir du 1er avril 2004, jusque
% fin 2008. Cette commande devient obsol\`ete en 2010.
% \begin{decl} |\RRdomaine| \arg{domaine} \end{decl}
% L'argument est un entier entre 1 et 5.
% Ces domaines sont \`a utiliser \`a partir de 2009. Le num\'ero
% apparait sur la page de couverture, et l'intitul\'e fran\c cais sur
% la page de titre int\'erieure. 
%  Ces commandes deviennent obsol\`ete en 2011.
% \begin{decl} |\RRthemeProj| \arg{proj} \\ 
%  |\RRdomaineProj|\arg{proj}
% \end{decl}
% Ces deux commandes sont fournies par le package |RRthemes|, elles
% prennent en argument un nom d'\'equipe-projet, et mettent sur la
% page de titre le th\`eme ou domaine associ\'es. N'utiliser qu'une
% seule des deux commandes. Le nom du projet est en lettres
% minuscules non accentu\'ees, il peut contenir des espaces.
% Si vous utilisez une de ces commandes,
% n'utiliser pas |\RRtheme| ni |\RRdomaine|.
%  Ces commandes deviennent obsol\`ete en 2011.
% \begin{decl} |\RRthemeProjBis| \arg{proj} \\ 
% |\RRdomaineProjBis| \arg{proj}
% \end{decl}
% Ces commandes permettent de sp\'ecifier un second th\`eme ou domaine
% sur la page de titre int\'erieure. 
%  Ces commandes deviennent obsol\`ete en 2011.
% \begin{decl} |\URLorraine| \\
%  |\URRennes| \\
%  |\URRhoneAlpes| \\
%  |\URRocq| \\
%  |\URFuturs| \\
%  |\URSophia| \end{decl}
% Utilisez une et une seule des commandes pr\'ec\'edentes. Si vous \^etes plusieurs
% auteurs dans des UR diff\'erentes, prenez-en une al\'eatoirement. Cette commande
% pr\'ecise quel logo mettre sur la page~1, elle donne l'adresse et le num\'ero de
% t\'el\'ephone de l'UR, et conditionne la 4\ieme{} page  de couverture.
%    \begin{macrocode}
%%
%% \URLorraine % pour ceux qui sont \`a l'est
%% \URRennes  % pour ceux qui sont \`a l'ouest
%% \URRhoneAlpes % pour ceux qui sont dans les montagnes
%% \URRocq % pour ceux qui sont au centre de la France
%% \URFuturs % pour ceux qui sont dans le virtuel
%% \URSophia % pour ceux qui sont au Sud.
%    \end{macrocode}
% \begin{decl} |\RCBordeaux| \\
%  |\RCLille| \\ |\RCSaclay| \\
%  |\RCGrenoble| \\
%  |\RCParis| \\ |\RCNancy| \\
%  |\RCRennes| \\
%  |\RCSophia| \end{decl}
% Vous pouvez utiliser les commandes du type |\URxx|, ou les nouvelles
% commandes de la forme |\RCyy| (o\`u yy est le premier mot du nom du Centre
% de Recherche, par exemple Nancy ou Sophia).
%    \begin{macrocode}
%%
%% \RCBordeaux % centre de recherche Bordeaux - Sud Ouest
%% \RCLille % centre de recherche Lille Nord Europe
%% \RCParis % Paris Rocquencourt
%% \RCSaclay % Saclay \^Ile de France
%% \RCGrenoble % Grenoble - Rh\^one-Alpes
%% \RCNancy % Nancy - Grand Est
%% \RCRennes % Rennes - Bretagne Atlantique
\RCSophia % Sophia Antipolis M\'editerran\'ee
%    \end{macrocode}

% Comme tout est dit, on peut commencer \`a travailler.
%    \begin{macrocode}
%%
\begin{document}
%    \end{macrocode}
% \begin{decl} |\makeRR| \\ |\makeRT| \end{decl}
% La commande |\makeRR| doit \^etre utilis\'ee juste apr\`es le |\begin{document}|
% dans le cas d'un rapport de recherche. S'il s'agit d'un rapport technique,
% on utilisera |\makeRT|.
%    \begin{macrocode}
%%
\makeRR   % cas d'un rapport de recherche
%% \makeRT % cas d'un rapport technique.
%% a partir d'ici, chacun fait comme il le souhaite
\section{First section}
Here is some text for the first section, and a label\label{sec1}. 
Uses version \RRfileversion\ of the package.\newpage
\section{Second section}
Text for the second section. This is closely related to the text in
section \ref{sec1} on page \pageref{sec1}. \newpage
\tableofcontents
%    \end{macrocode}

% On termine le document comme d'habitude, via |\end{document}|.
%    \begin{macrocode}
\end{document}
%</sample>
%    \end{macrocode}
% \section{Les th\`emes et domaines des projets}
% Le fichier |RRtheme.sty|  a \'et\'e cr\'e\'e \`a partir du fichier XML
% contenant les projets du Rapport d'activit\'e 2009. Si votre projet
% a \'et\'e cr\'e\'e apr\`es la date de mise \`a jour de ce fichier
% (par exemple, Athena en janvier 2010), il ne figure pas dans la
% liste, et c'est \`a vous de le rajouter. 
% 
% Obsolete en 2011.
%
% \section{Description du code du package RR}
% \subsection{Introduction}
% \changes{v3.7i}{2001/07/18}{T1/OT1 switch was wrong}
% \changes{v4.3}{2006/05/19}{Made options obsolete.}
%    \begin{macrocode}
%<*RR>
%    \end{macrocode}
% Il y a 4 options obsol\`etes qui provoquent une erreur. La
% derni\`ere option positionne un bool\'een.
%    \begin{macrocode}
\typeout{Style de documents Rapport de recherche Inria : version
\RRfileversion\space du \RRfiledate}
\newif\ifRR@french \RR@frenchfalse
\newif\if@dc@french \@dc@frenchfalse
\def\RR@badoption#1{\@latex@error
   {Option `#1' removed, see documentation}\@eha}
\DeclareOption{french}{\@dc@frenchtrue}
\DeclareOption{T1}{\RR@badoption{T1}}
\DeclareOption{OT1}{\RR@badoption{OT1}}
\DeclareOption{noinputenc}{\RR@badoption{noinputenc}}
\DeclareOption{utf8}{\RR@badoption{utf8}}
\ProcessOptions\relax
%    \end{macrocode}
% \changes{v4.1}{2004/11/12}{package ifpdf added}
% \changes{v4.1}{2004/11/12}{degree command removed}
% \changes{v4.3}{2006/05/03}{package ifpdf removed as unused. Removed encoding stuff.}
% \changes{v4.3}{2006/05/03}{Removed encoding stuff.}
% \changes{v3.7g}{2000/12/13}{Test to pdfoutput changed}
% \changes{v3.7h}{2001/07/11}{added dvips option to graphicx (garavel)}
% \changes{v3.8c}{2003/04/10}{extension ps}
% \changes{v4.1}{2004/11/12}{textcomp package always loaded}
% \changes{v4.1}{2004/11/12}{fontenc loading simplified}
% \changes{v4.3}{2006/05/03}{UTF support added}
% \changes{v4.3}{2006/05/03}{T1 independent of pdflatex}
% \changes{v4.3}{2006/05/03}{Removed input encoding at all}
% \changes{v4.4}{2006/05/23}{Removed output encoding also}
% On charge les packages de fontes et d'encodages.
% En ce qui concerne les packages d'encodage entr\'ee-sortie, on ne garde plus
% que |textcomp|, pour avoir le symbole num\'ero.
%    \begin{macrocode}
\RequirePackage{textcomp}\RequirePackage{color}
%    \end{macrocode}
% Chargement du package graphicx.
% \changes{v4.3}{2006/05/03}{Autodetection of graphics extension, logoext removed}
% \changes{v4.1}{2004/11/12}{Use ifpdf}
% \changes{v4.1}{2004/11/12}{Graphics loaded without options}
%    \begin{macrocode}
\RequirePackage{graphicx}
%    \end{macrocode}
% \begin{macro}{twoside}
% \changes{v4.11}{2010/02/01}{@Twosidetrue commented out}
% La ligne suivante a \'et\'e mise en commentaire en Janvier 2010. Utlisez
% l'option de classe |twoside| \`a la place.
%    \begin{macrocode}
%% \@twosidetrue
%    \end{macrocode}
% \end{macro}
% Les fontes Postscript :
% \begin{macro}{\title@font}
% \begin{macro}{\rrno@font}
% \begin{macro}{\author@font}
% \begin{macro}{\date@font}
% \begin{macro}{\eighttm}
% \begin{macro}{\theme@font}
% \begin{macro}{\issn@font}
% \begin{macro}{\cr@font}
% Ces macros avaient des noms un peu court. 
% |twenty| = 20, |fifteen|=15, |twelve| = 12,
% |ten| = 10, |eight| = 8, et pour les suffixes, |tb| est times bold, |tr| est
% times roman, |hv| est helvetica.
% \changes{v3.7b}{1998/02/11}{macro twentytb changed from 20pt to 20.74}
% \changes{v3.7b}{1998/02/<11}{Changement des noms de fonte postscript}
% \changes{v4.10}{2007/12/13}{Changement de nom de certaines fontes}
% \changes{v5.0}{2011/09/28}{Changement de nom/valeur de certaines fontes}
% \changes{v5.1}{2011/10/21}{cr@font added}
% \changes{v5.1}{2011/10/24}{fomnt naming simplified}
%    \begin{macrocode}
\def\@psenc{T1}
\newcommand\times@font{\fontencoding{T1}\fontfamily{ptm}}
\newcommand\helvetica@font{\fontencoding{T1}\fontfamily{phv}}
\newcommand\title@font{\times@font\fontseries{b}%
   \fontshape{n}\fontsize{36pt}{40pt}\selectfont}
\newcommand\rrno@font{\times@font\fontseries{b}%
   \fontshape{n}\fontsize{15pt}{20pt}\selectfont}
\newcommand\author@font{\times@font\fontseries{b}%
   \fontshape{n}\fontsize{13.3pt}{16pt}\selectfont}
\newcommand\theme@font{\times@font\fontseries{m}%
   \fontshape{n}\fontsize{10pt}{11pt}\selectfont}
\newcommand\issn@font{\helvetica@font\fontseries{m}%
   \fontshape{n}\fontsize{8pt}{10pt}\selectfont}
\newcommand\cr@font{\helvetica@font\fontseries{b}%
   \fontshape{n}\fontsize{7.5pt}{10pt}\selectfont}
%    \end{macrocode}
% \end{macro}
% \end{macro}
% \end{macro}
% \end{macro}
% \end{macro}
% \end{macro}
% \end{macro}
% \end{macro}
% \subsection{atxy}
% Code inspir\'e du package atxy. 
%    \begin{macrocode}
\newbox\RR@atxybox%
\newif\ifRR@atxy\RR@atxyfalse
%    \end{macrocode}
% syntaxe : |\RR@atxy{hpos}{vpos}{text}|.
% Positionne le texte en absolu l\`a o\`u on le demande. On commence par mettre
% dans une bo\^\i te tous les bouts de texte.
%
%    \begin{macrocode}
\newcommand\RR@atxy[3]{\global\setbox\RR@atxybox=\hbox
 {\unhbox\RR@atxybox
  \vtop to 0pt{\kern #2\hbox to 0pt{\kern #1\relax #3\hss}\vss}}%
 \global\RR@atxytrue}
%    \end{macrocode}
% C'est maintenant que les choses se gatent. Il faut positionner la bo\^\i te. 
%    \begin{macrocode}
\def\@userratxy{%
 \ifRR@atxy{%
  \if@twoside
    \ifodd\count\z@
         \let\@themargin\oddsidemargin
    \else \let\@themargin\evensidemargin
    \fi
  \fi
  \vtop to 0pt{\kern-\headsep \kern-\topmargin \kern-\headheight 
               \kern-1in \kern-\voffset
               \hbox to 0pt{\kern-\@themargin \kern-1in \kern-\hoffset
                            \unhbox\RR@atxybox \hss}\vss}%
 }\fi
 \global\RR@atxyfalse}%
%    \end{macrocode}
% On red\'efinit |\output| pour faire fonctionner la m\'ecanique.
% \changes{v5.1}{2011/11/08}{AtbeginDocument for stfloats (Charles Andre)}
%    \begin{macrocode}x1
\AtBeginDocument{\toks0={\setbox255=\vbox{\@userratxy \unvbox255}}
\output=\expandafter{\the\toks0\the\output}}
%    \end{macrocode}
% \subsection{Commandes de base}
% \begin{macro}{\titlehead}
% \begin{macro}{\@titlehead}
% \begin{macro}{\authorhead}
% \begin{macro}{\@authorhead}
% Les en-t\^etes de pages utilisent les valeurs de |\@titlehead| et
% |\@authorhead|. Ces valeurs sont positionn\'ees par les macros |\RRauthor| et
% |\RRtitle|. 
% L'utilisateur peut les modifier via |\titlehead| et
% |\authorhead| (cas o\`u le titre est trop long, cas o\`u il y a trop d'auteurs).
% \changes{v3.7h}{2001/06/06}{titlehead undef'ed, Lasgouttes }
%    \begin{macrocode}
\let\titlehead=\undefined
\newcommand\titlehead[1]{\gdef\@titlehead{#1}}
\newcommand\authorhead[1]{\gdef\@authorhead{#1}}
\titlehead{}
\authorhead{}
%    \end{macrocode}
% \end{macro}
% \end{macro}
% \end{macro}
% \end{macro}
% \begin{macro}{\RRNo}
% \begin{macro}{\RR@No}
% La macro |\RRNo| est utilis\'ee pour positionner le num\'ero du rapport.
% Il fallait autrefois que la valeur de |\RR@No| contienne 4 points
% d'interrogations successifs.
% \changes{v3.7b}{1998/03/12}{RRno correct, meme si vide}
%    \begin{macrocode}
\newcommand\RRNo[1]{\gdef\RR@No{#1}}
\newcommand\RR@No{????}
%    \end{macrocode}
% \end{macro}
% \end{macro}
% \begin{macro}{\RRdate}
% \begin{macro}{\RR@date}
% \begin{macro}{\RR@month}
% La macro |\RR@date| contient la date \`a mettre sur le rapport. Par d\'efaut, on
% met la date courante. L'utilisateur peut positionner une date via
% |\RRdate|. La date par d\'efaut est en fran\c cais.
% \changes{v3.4}{1997/09/22}{Manquait novembre dans la liste des mois}
% \changes{v3.5}{1997/11/24}{Message imprime une seule fois}
% \changes{v4.9}{2007/11/14}{Use protected@xdef}
%    \begin{macrocode}
\newcommand\RR@month{\ifcase\the\month
 \or Janvier\or F\'evrier\or Mars\or Avril\or Mai\or Juin \or Juillet%
 \or Ao\^ut\or Septembre\or Octobre\or Novembre\or D\'ecembre\fi}
\newcommand\RRdate[1]{\gdef\RR@date{#1}}
\newcommand\RR@date{%
 \protected@xdef\RR@date{\RR@month\space \number\the\year}%
 \RR@date%
 \@warning{No \string\RRdate \space seen:
         \RR@date\space will be used.}}
%    \end{macrocode}
% \end{macro}
% \end{macro}
% \end{macro}
% \begin{macro}{\RRdater}
% \begin{macro}{\RR@dater}
% La macro |\RR@dater| est utilis\'ee dans le cas d'une nouvelle version :
% La date originale est dans |\RR@date|, la date de r\'evision dans 
% |\RR@dater|.  Par d\'efaut, on
% met la date courante. L'utilisateur peut positionner une date via
% |\RRdater|. 
% \changes{v4.5}{2006/11/14}{Macro added}
% \changes{v4.9}{2007/11/14}{Use protected@xdef}
%    \begin{macrocode}
\newcommand\RRdater[1]{\gdef\RR@dater{#1}}
\newcommand\RR@dater{%
 \protected@xdef\RR@dater{\RR@month\space \number\the\year}%
 \RR@dater%
 \@warning{No \string\RRdater \space seen:
         \RR@dater\space will be used.}}
%    \end{macrocode}
% \end{macro}
% \end{macro}
% \begin{macro}{\RRversion}
% \begin{macro}{\RR@version}
% \begin{macro}{\ifRR@version}
% La macro |\RR@version| contient le num\'ero de version, si ce n'est pas la 
% version initiale. Le bool\'een |\ifRR@version| dit s'il s'agit de 
% la version initiale ou une r\'evision. La commande |\RRversion| positionne
% la variable interne et le bool\'een.
% \changes{v4.5}{2006/11/14}{Macro added}
%    \begin{macrocode}
\newcommand\RRversion[1]{\gdef\RR@version{#1}\global\RR@versiontrue}
\newcommand\RR@version{Initial version}%
\newif\ifRR@version\RR@versionfalse
%    \end{macrocode}
% \end{macro}
% \end{macro}
% \end{macro}
% \begin{macro}{\RRauthor}
% \begin{macro}{\RR@author}
% \begin{macro}{\and}
% La macro |\RRauthor| est utilis\'ee pour les noms des auteurs du rapport. On
% s\'epare plusieurs auteurs par |\and|. Cette macro sauve le r\'esultat dans
% |\RR@author|. Elle positionne aussi |\@authorhead|, pour les en-t\^etes de
% page. 
% \changes{v3.4}{1997/09/17}{Def de la macro and sortie de author}
%    \begin{macrocode}
\newcommand\RRauthor[1]{\gdef\RR@author{#1}\gdef\@authorhead{#1}}
\newcommand\RR@author{??\gdef\RR@author{??}%
\@latex@error{No author given,^^J
   use \string\RRauthor\string{Prenom1 Nom1 \string\and\space  Prenom2
    Nom2 \string\and \space...\string}}\@eha}
%    \end{macrocode}
% \end{macro}
% \end{macro}
% \end{macro}
% \begin{macro}{\RRtitle}
% \begin{macro}{\RR@title}
% \begin{macro}{\RRetitle}
% \begin{macro}{\RR@etitle}
% Les deux macros |\RRtitle| et |\RRetitle| prennent en argument le titre en
% fran\c cais et en anglais, respectivement. Elles mettent cet argument dans une
% macro interne de m\^eme nom.
%    \begin{macrocode}
\newcommand\RRtitle[1]{\gdef\RR@title{#1}}
\newcommand\RRetitle[1]{\gdef\RR@etitle{#1}}
%    \end{macrocode}
% Le code qui suit essaie de trouver dans quelle langue est le document.
% On essaie de savoir si la langue fran\c caise est d\'efinie pour babel,
% ou alors si le package |french| est charg\'e. Une fois fait ceci, on regarde
% si |\titlehead| a \'et\'e utilis\'e ; si non, on utilise le titre dans la
% langue principale.
%    \begin{macrocode}
\def\set@titlehead{%
  {\if@dc@french\global\RR@frenchtrue
   \else
     \def\tempa{french}\def\tempb{frenchb}%
     \ifx\bbl@main@language\tempa\global\RR@frenchtrue\fi
     \ifx\bbl@main@language\tempb\global\RR@frenchtrue\fi
     \@ifundefined{ifFrench}{}{\global\RR@frenchtrue}%
   \fi
   \def\tempa{}%
   \ifx\@titlehead\tempa \ifRR@french \gdef\@titlehead{\RR@title} \else
     \gdef\@titlehead{\RR@etitle}\fi\fi}}
%    \end{macrocode}
% Le code plus haut est ex\'ecut\'e au d\'ebut du document. Comme cela le
% chargement du fichier |babel| ou |french| peut \^etre fait n'importe
% quand. On met un titre par d\'efaut.
%    \begin{macrocode}
\AtBeginDocument{\set@titlehead}
\newcommand\RR@title{??\@latex@error{French title missing, use 
  \string\RRtitle.}\@eha}
\newcommand\RR@etitle{??\@latex@error{English title missing, use 
  \string\RRetitle.}\@eha}
%    \end{macrocode}
% \end{macro}
% \end{macro}
% \end{macro}
% \end{macro}
% \begin{macro}{\RRnote}
% \begin{macro}{\RR@note}
% La macro |\RRnote| permet de faire des notes sur la page 1. On collectionne
% dans |\RR@note| l'ensemble de ces textes, en disant que c'est des notes de
% bas de page.
% \changes{v3.4}{1997/09/17}{Utilisation de unexpandableprotect}
% \changes{v4.9}{2007/11/14}{Use protected@xdef}
%    \begin{macrocode}
\newcommand\RRnote[1]{\begingroup
  \let\protect\@unexpandable@protect
  \protected@xdef\RR@note{\RR@note \protect\footnotetext[0]{#1\par}}%
  \endgroup}
\newcommand\RR@note{}
%    \end{macrocode}
% \end{macro}
% \end{macro}
% \begin{macro}{\RRprogramme}
% Cette macro ne sert plus, il n'y a plus de programme.
% \changes{v3.4}{1997/09/25}{Suppression de RRprogramme}
% Il y avait une macro |\n@mth@mC| qui faisait la m\^eme chose, mais elle a \'et\'e supprim\'ee.
% \changes{v3.4}{1997/09/17}{Suppression de n@mth@mC}
% \changes{v4.0}{2004/04/22}{Suppression de n@mth@m}
% \changes{v4.0}{2004/04/22}{Suppression de compteur@theme}
% \changes{v4.0}{2004/04/22}{Suppression de @compteur@theme@un}
% \changes{v4.0}{2004/04/22}{Suppression de @compteur@theme@deux}
% \changes{v4.0}{2004/04/22}{Suppression de n@mth@m}
% \changes{v4.0}{2004/04/22}{Suppression de n@mth@m}
% \changes{v3.4}{1997/09/24}{Suppression de if@monothem}
% \changes{v4.3}{2006/05/03}{Suppression de RR@THMTYPE}
% \changes{v4.3}{2006/05/19}{Warnings transformed into errors}
% \changes{v4.11}{2009/06/09}{Domaine au lieu de theme}
% \changes{v5.0}{2011/09/28}{Suppression des domaines}
% \end{macro}
% \begin{macro}{\RR@obsolete}
% \begin{macro}{\RR@domnum}
% \begin{macro}{\RR@domname}
% \begin{macro}{\RR@domtype}
% On met dans |\RR@domnum| le num\'ero du domaine, dans |\RR@domname| le nom du
% domaine. On utilise deux macros qui contiennent le mot \og domaine
% \fg. Changements 2010 : La premi\`ere commande contient non plus un
% num\'ero, mais le texte \`a mettre sur la page de couverture, et la
% seconde commande le texte sur la page int\'erieure. Le style
% |RRthemes| remplace le nom de \og domaine
% \fg\ par un nom anglais.  Changements 2011 : tout obsol\`ete. 
%    \begin{macrocode}
\def\RRobsolete#1{%
\@latex@error{Obsolete command `\@backslashchar#1' ignored}\@eha}
%    \end{macrocode}
% \end{macro}
% \end{macro}
% \end{macro}
% \end{macro}

% \begin{macro}{\RRdomaine}
% \begin{macro}{\RRtheme}
% \begin{macro}{\RRdomaineProj}
% \begin{macro}{\RRthemeProj}
% \begin{macro}{\RRdomaineProjBis}
% \begin{macro}{\RRthemeProjBis}
% Toute cela est obsol\`ete en 2011 
% \changes{v4.0}{2004/04/22}{Handling of themes modified}
% \changes{v4.0}{2004/04/22}{En majuscules}
% \changes{v4.11}{2009/06/09}{Domaine au lieu de theme}
% \changes{v5.0}{2011/09/28}{Simplification de RRtheme}
%    \begin{macrocode}
\newcommand\RRtheme[1]{\RRobsolete{RRtheme}}
\newcommand\RRdomaine[1]{\RRobsolete{RRdomaine}}
\newcommand\RRthemeProj[1]{\RRobsolete{RRthemeProj}}
\newcommand\RRdomaineProj[1]{\RRobsolete{RRdomaineProj}}
\newcommand\RRthemeProjBis[1]{\RRobsolete{RRthemeProjBis}}
\newcommand\RRdomaineProjBis[1]{\RRobsolete{RRdomaineProjBis}}
%    \end{macrocode}
% \end{macro}
% \end{macro}
% \end{macro}
% \end{macro}
% \end{macro}
% \end{macro}

% \begin{macro}{\eval@theme}
% \begin{macro}{\THCom}
% \begin{macro}{\THCog}
% \begin{macro}{\THSym}
% \begin{macro}{\THNum}
% \begin{macro}{\THBio}
% On d\'efinit ici les valeurs des 5 domaines (ne sert plus)
%    \begin{macrocode}
%% dom1{Math\'ematiques appliqu\'ees, calcul et simulation}
%% dom2{Algorithmique, programmation, logiciels et architectures}
%% dom3{R\'eseaux, syst\`emes et services, calcul distribu\'e}
%% dom4{Perception, cognition, interaction}
%% dom5{STIC pour les sciences de la vie et de l'environnement} 
%    \end{macrocode}
% \end{macro}
% \end{macro}
% \end{macro}
% \end{macro}
% \end{macro}
% \end{macro}
% \begin{macro}{\TH@add}
% \changes{v4.9}{2007/11/14}{Use protected@xdef}
% \changes{v4.11}{2009/06/09}{suppression de la commande}
% \changes{v5.0}{2011/09/28}{suppression de la commande}
% Cas de plusieurs th\`emes. Il y avait de la place sur la page de
% garde pour mettre \og Th\`emes 1 et 2 \fg; lorsque les num\'eros ont
% disparu, la place a commenc\'e  \`a manquer, et on a d\'efini une commande
% qui n'affiche que le premier. Cette commande ne sert plus car il n'y a plus
% de th\`emes.
% \end{macro}
% \begin{macro}{\RRprojet}
% \begin{macro}{\RRprojets}
% \begin{macro}{\RR@projet}
% \begin{macro}{\RR@prjtype}
% \changes{v4.8}{2007/06/25}{Projet replaced by Equipe-Projet}
% \changes{v4.10}{2007/12/13}{Alternate names added}
% \changes{v5.0}{20011/09/28}{Team/language}
% La macro |\RRprojet| permet de positionner le nom du projet dans
% |\RR@projet|. S'il y a deux projets, il faut utiliser |\RRprojets|.
% On met le mot \og projet \fg, ou \og projets \fg\ dans |\RR@prjtype|. Le nom
% officiel est maintenant \'equipe-projet ou Project-Team.
%    \begin{macrocode}
\newcommand\RR@prjtype{Project-Team}
\newcommand\RR@fprjtype{\'Equipe-Projet}
\newcommand\RRprojet[1]{\gdef\RR@projet{#1}}
\newcommand\RRprojets[1]{\gdef\RR@fprjtype{\'Equipes-Projets}
  \gdef\RR@prjtype{Project-Teams}\gdef\RR@projet{#1}}
\let\RRequipe\RRprojet
\let\RRequipes\RRprojets
\newcommand\RR@projet{??\latex@error{Team missing, use
 \string\RRprojet}\@eha}
%    \end{macrocode}
% \end{macro}
% \end{macro}
% \end{macro}
% \end{macro}
% \begin{macro}{\RRnbpage}
% \begin{macro}{\RR@nbpage}
% \changes{v3.4}{1997/09/17}{Utilisation de pageref}
% Normalement, \LaTeX\ compte bien le nombre de pages, mais pour le cas o\`u, 
% on autorise l'utilisateur \`a mettre le nombre de pages dans |\RRnbpage|. Le
% nombre de pages est la valeur de la derni\`ere page du rapport.
%    \begin{macrocode}
\newcommand\RRnbpage[1]{\gdef\RR@nbpage{#1}}
\newcommand\RR@nbpage{\@ifundefined{r@RRlastpageofreport}{??}
  {\pageref{RRlastpageofreport}}}
%    \end{macrocode}
% \end{macro}
% \end{macro}
% \begin{macro}{\RR@resume}
% \begin{macro}{\RR@abstract}
% \begin{macro}{\RRresume}
% \begin{macro}{\RRabstract}
% Le r\'esum\'e en fran\c cais et en anglais se font via les commandes |\RRresume| et
% |\RRabstract|, qui mettent dans une variable cach\'ee le texte. 
% On donne une valeur par d\'efaut \`a ces choses. 
% \changes{v3.7b}{1998/02/11}{Real name is RRabstract, not abstract}
% \changes{v5.0}{2011/09/28}{Language switch removed removed}
%    \begin{macrocode}
\newcommand\RRresume[1]{%
    \long\def\RR@resume{\noindent{\bf R\'esum\'e : } #1\par}}
\newcommand\RRabstract[1]{%
    \long\def\RR@abstract{\noindent{\bf Abstract: } #1\par}}
\RRresume{Pas de r\'esum\'e}
\RRabstract{No abstract}
%    \end{macrocode}
% \end{macro}
% \end{macro}
% \end{macro}
% \end{macro}
% \begin{macro}{\RR@motcle}
% \begin{macro}{\RR@keyword}
% \begin{macro}{\RRmotcle}
% \begin{macro}{\RRkeyword}
% M\^eme technique pour les mots cl\'es.
% \changes{v3.4}{1997/09/25}{fhyph mal place dans keyword}
% \changes{v3.5}{1997/10/05}{s a mots-cles}
% \changes{v5.0}{2011/09/28}{Language switch removed removed}
%    \begin{macrocode}
\newcommand\RRmotcle[1]{%
   \def\RR@motcle{\noindent{\bf Mots-cl\'es : } #1\par}}
\newcommand\RRkeyword[1]{%
   \def\RR@keyword{\noindent{\bf Keywords: } #1\par}}
\RRmotcle{Pas de motclef}
\RRkeyword{No keywords}
%    \end{macrocode}
% \end{macro}
% \end{macro}
% \end{macro}
% \end{macro}
% Des synonymes, sans le pr\'efixe \og RR \fg.
%    \begin{macrocode}
\let\resume=\RRresume \let\abstract=\RRabstract
\let\motcle=\RRmotcle \let\keyword=\RRkeyword
%    \end{macrocode}
%\subsection{Les centres de recherche}
% \begin{macro}{\RR@rc@name}
% \begin{macro}{\RR@CRnum}
% \begin{macro}{\@title@logo@name}
% \begin{macro}{\RR@CRaddress}
% \changes{v3.4}{1997/09/24}{Supression de RR@CRname}
% On met le num\'ero de l'UR dans |\RR@CRnum|, le nom
% du fichier ps qui contient le logo dans |\@title@logo@name|, et l'adresse dans
% |\RR@CRaddress|. Dans la version actuelle, on utilise un nom \`a la plce
% de chiffres dans |\RR@rc@name|.
%    \begin{macrocode}
\let\RR@rc@name\relax \let\RR@rc\empty
%    \end{macrocode}
% \end{macro}\end{macro}\end{macro}\end{macro}
% \begin{macro}{\URLorraine}
% \begin{macro}{\RCNancy}
% \changes{v3.7c}{1998/04/28}{Remplacement de telecopie par fax}
% \changes{v3.7c}{1998/04/28}{Tel lorraine sur 1 ligne}
% \changes{v3.7f}{1999/09/17}{deux num\'eros de t\'el\'ephone}
% \changes{v3.7f}{1999/09/17}{Suppression de l'antenne de Metz}
% \changes{v4.8}{2007/06/25}{Unit\'e replaced by Centre}
% \changes{v4.10}{2007/12/13}{New logo in RR@rc@name}
% \changes{v5.0}{2011/09/28}{Adress removed}
% \changes{v5.1}{2011/10/21}{Adress added again}
% On commence par la Lorraine (ordre alphab\'etique). Le classement
% alphab\'etique par ville est maintenant : Bordeaux, Grenoble, Lille, Nancy,
% Paris, Saclay, Sophia. 
%    \begin{macrocode}
\newcommand\RCNancy{\gdef\RR@rc@name{nancy}
  \def\RR@rc{NANCY -- GRAND EST}
  \def\RR@adr{615 rue du Jardin Botanique \\CS20101\\
    54603 Villers-l\`es-Nancy Cedex}}
\let\URLorraine\RCNancy
%    \end{macrocode}
% \end{macro}\end{macro}
% \begin{macro}{\URRennes}
% \begin{macro}{\RCRennes}
% Le suivant est Rennes. Pour des raisons historiques, c'est le d\'efaut.
% \changes{v3.7f}{1999/09/17}{deux num\'eros de t\'el\'ephone}
% \changes{v5.0}{2011/09/28}{Adress removed}
% \changes{v5.1}{2011/10/21}{Adress added again}
%    \begin{macrocode}
\newcommand\RCRennes{\gdef\RR@rc@name{rennes}
  \def\RR@rc{RENNES -- BRETAGNE ATLANTIQUE}
  \def\RR@adr{Campus universitaire de Beaulieu\\
    35042 Rennes Cedex}}
\let\URRennes\RCRennes
% \URRennes
%    \end{macrocode}
% \end{macro}\end{macro}
% \begin{macro}{\URRhoneAlpes}
% \begin{macro}{\RCGrenoble}
% \changes{v5.0}{2011/09/28}{Adress removed}
% \changes{v5.1}{2011/10/21}{Adress added again}
% Puis Grenoble
%    \begin{macrocode}
\newcommand\RCGrenoble{
  \gdef\RR@rc@name{grenoble}\def\RR@rc{GRENOBLE -- RH\^ONE-ALPES}
  \def\RR@adr{Inovall\'ee\\655 avenue de l'Europe Montbonnot\\
    38334  Saint Ismier Cedex}}
\let\URRhoneAlpes\RCGrenoble
%    \end{macrocode}
% \end{macro}\end{macro}
% \begin{macro}{\URRocq}
% \begin{macro}{\RCParis}
% Ne pas oublier Rocquencourt.
% \changes{v3.7f}{1999/09/17}{deux num\'eros de t\'el\'ephone}
% \changes{v5.0}{2011/09/28}{Adress removed}
% \changes{v5.1}{2011/10/21}{Adress added again}
%    \begin{macrocode}
\newcommand\RCParis{\gdef\RR@rc@name{paris}
  \def\RR@rc{PARIS -- ROCQUENCOURT}
  \def\RR@adr{Domaine de Voluceau, - Rocquencourt\\
     B.P. 105 - 78153 Le Chesnay Cedex}}
\let\URRocq\RCParis
%    \end{macrocode}
% \end{macro}\end{macro}
% \begin{macro}{\URSophia}
% \begin{macro}{\RCSophia}
% On termine par Sophia Antipolis.
% \changes{v3.6}{1997/12/17}{date de changement de numero}
% \changes{v3.7}{1998/01/19}{Suppression de l'ancien numero}
% \changes{v3.7f}{1999/09/17}{Suppression des points dans BP}
% \changes{v3.7f}{1999/09/17}{deux num\'eros de t\'el\'ephone}
% \changes{v5.0}{2011/09/28}{Adress removed}
% \changes{v5.1}{2011/10/21}{Adress added again}
%    \begin{macrocode}
\newcommand\RCSophia{\gdef\RR@rc@name{sophia}
  \def\RR@rc{SOPHIA ANTIPOLIS -- M\'EDITERRAN\'EE}
  \def\RR@adr{2004 route des Lucioles - BP 93\\
    06902 Sophia Antipolis Cedex}}
\let\URSophia\RCSophia\RCSophia
%    \end{macrocode}
% \end{macro}\end{macro}
% \begin{macro}{\URFuturs}
% Unit\'e de recherche de Futurs : s'est partag\'ee en trois centres de
% recherche le 1er janvier 2008.
% \changes{v3.8b}{2002/11/07}{Ajout de futurs}
% \changes{v3.8c}{2003/04/10}{nouvelle adresse}
% \changes{v4.3}{2006/05/03}{numero de FAX}
% \changes{v4.3}{2006/05/19}{Futurs is the fallback UR}
% \changes{v5.0}{2011/09/28}{Futurs is obsolete}
% \changes{v5.0}{2011/09/28}{Adress removed}
%    \begin{macrocode}
\newcommand\URFuturs{\RRobsolete{URFuturs}}
%    \end{macrocode}
% \end{macro}
% \begin{macro}{\RCBordeaux}
% Bordeaux : une partie de Futurs.
% \changes{v4.9a}{2007/11/20}{macro added}
% \changes{v5.0}{2011/09/28}{Adress removed}
% \changes{v5.1}{2011/10/21}{Adress added again}
%    \begin{macrocode}
\newcommand\RCBordeaux{\gdef\RR@rc@name{bordeaux}
  \def\RR@rc{BORDEAUX -- SUD-OUEST}
  \def\RR@adr{351, Cours de la Lib\'eration\\ B\^atiment A 29\\
     33405 Talence Cedex}}
%    \end{macrocode}
% \end{macro}
% \begin{macro}{\RCLille}
% Lille : une seconde partie de Futurs.
% \changes{v4.9a}{2007/11/20}{macro added}
% \changes{v5.0}{2011/09/28}{Adress removed}
% \changes{v5.1}{2011/10/21}{Adress added again}
%    \begin{macrocode}
\newcommand\RCLille{\gdef\RR@rc@name{lille}
  \def\RR@rc{LILLE -- NORD EUROPE}
  \def\RR@adr{Parc scientifique de la Haute-Borne\\
   40 avenue Halley - B\^at A - Park Plaza\\
   59650 Villeneuve d'Ascq}}
%    \end{macrocode}
% \end{macro}
% \begin{macro}{\RCSaclay}
%  Saclay : la troisi\`eme partie de Futurs
% \changes{v4.9a}{2007/11/20}{macro added}
% \changes{v4.9a}{2007/11/22}{Hyphens added}
% \changes{v5.0}{2011/09/28}{Adress removed}
% \changes{v5.1}{2011/10/21}{Adress added again}
%    \begin{macrocode}
\newcommand\RCSaclay{\gdef\RR@rc@name{saclay}
   \def\RR@rc{SACLAY -- \^ILE-DE-FRANCE}
   \def\RR@adr{Parc Orsay Universit\'e \\4 rue Jacques Monod\\
      91893 Orsay Cedex}}
%    \end{macrocode}
% \end{macro}
% \subsection{Derni\`ere page}
% \begin{macro}{\lastRRpage}
% \begin{macro}{\rr@frame}
% La page de titre int\'erieure et la derni\`ere page contiennent une image de
% fond avec l'adresse du centre. 
% On utilise |\AtEndDocument| pour ex\'ecuter le code de |\lastRRpage|.
% On cr\'ee une page vide, sans en-t\^etes, avec un label, pour pouvoir
% compter le nombre de pages. On met la m\^eme image de fond que sur la page
% de titre int\'erieure et le logo de l'\'editeur avec l'ISSN.
% \changes{v3.7h}{2001/06/14}{clearpage, gouz\'e}
% \changes{v5.0}{2011/09/28}{Simplification derniere page}
% \changes{v5.0}{2011/09/28}{Use phantomsection for number of pages}
% \changes{v3.4}{1997/10/19}{Utilisation de AtEndDocument}
% \changes{v5.1}{2011/10/21}{Publisher in english}
%    \begin{macrocode}
\newcommand\rr@frame{%
 \RR@atxy{0mm}{10mm}{\includegraphics{pagei}}
 \RR@atxy{38mm}{55mm}{\includegraphics[width=50mm]{logo-inria}}
 \RR@atxy{40mm}{256mm}{\hbox{\cr@font\color{red}RESEARCH CENTRE}}
 \RR@atxy{40mm}{259mm}{\hbox{\cr@font\color{red}\RR@rc}}
 \RR@atxy{40mm}{266mm}{\vbox{\hsize=18cm{\noindent\issn@font\RR@adr}}}}
\newcommand\lastRRpage{%
  \clearpage\thispagestyle{empty}%
  \csname phantomsection\endcsname
  \label{RRlastpageofreport}
  \addtocounter{page}{-1}
  \null\vfill\rr@frame
  \RR@atxy{98mm}{255mm}{\hbox{\vbox{\parindent=0pt \issn@font%
     \hsize=48mm \raggedright \hyphenpenalty 10000
      Publisher\\ Inria \\ Domaine de Voluceau - Rocquencourt\\
      BP 105 - 78153 Le Chesnay Cedex\\inria.fr\\[3mm]
      ISSN 0249-\RR@cond{6399}{0803}}}}
  \clearpage}
\AtEndDocument{\lastRRpage}
%    \end{macrocode}
% \end{macro}
% \end{macro}
% \begin{macro}{\if@mustprint}
% \begin{macro}{\@comspace}
% En 2011, la derni\`ere page \'etait plus compliqu\'ee: on y mettait
% les adresses de tous les CR, en premier le CR de l'auteur, puis les autres,
% par ordre alphab\'etique. \`A chaque CR est 
% associ\'e un num\'ero, on imprime dans l'ordre des num\'eros croissants. On
% utilise le bool\'een |\if@mustprint| s'il faut imprimer le CR. Autre
% subtilit\'e 
% : quand on imprime le premier CR, un met un changement de ligne apr\`es le
% nom, dans le cas contraire on met un deux-points. Le changement de ligne est
% |\\|, et la macro |\@comspace| imprime le deux-points.
%    \begin{macrocode}
%\newcommand\@comspace{ : }
%\newif\if@mustprint
%    \end{macrocode}
% \end{macro}
% \end{macro}
% \begin{macro}{\@printCR}
% \begin{macro}{\RR@Fr}
% \changes{v4.10}{2007/12/14}{Name changed from @printUR to @printCR}
% La macro |\@printCR| prend deux arguments $n$ et $i$. Elle est appell\'ee 12
% fois, d'abord avec $i=1$, puis avec $i=0$. La valeur de $n$ est 1, 2, 3, 4,
% 5, et 6. Le num\'ero 0 a \'et\'e affect\'e \`a l'UR Futurs, et ensuite les
% num\'eros 7, 8, et 9 aux trois CR issus de Futurs.  Le chiffre 12 indiqu\'e
% plus haut varie ainsi avec le temps ; si un nouveau centre est cr\'e\'e, il
% faudra remplacer |\if| par |\ifnum|.
% Dans le cas $i=1$, on veut imprimer le CR courant. 
% Dans le cas $i=0$,
% on veut imprimer les autres CR. Le premier argument est le num\'ero du CR.
%    \begin{macrocode}
%\newcommand\@printCR[2]{%
% \@mustprintfalse
% \ifx#21
%   \if #1\RR@CRnum \@mustprinttrue\fi
% \else\if #1\RR@CRnum \else\@mustprinttrue\fi\fi
% \if@mustprint
%    \end{macrocode}
% Dans le cas $i=1$, on utilise un corps 10 points, et sinon un corps 8
% points. La charte graphique demande un interligne de 9.5 points. On utilise
% un interligne de 10 (pour \'eviter de cr\'eer une fonte). On positionne la
% macro |\RR@optret|. Dans le cas $i=1$, c'est un changement de ligne, dans le
% cas contraire, c'est un deux-points. Pour ne pas trop radoter, on ne dit
% qu'une seule fois, via |\RR@Fr| que le centre est localis\'e en France.
% De plus, on met un peut plus d'espace vertical apr\`es le premier CR. 
% \changes{v3.7f}{1999/09/17}{mise \`a jour des adresses}
% \changes{v3.8c}{2003/04/10}{nouvelle adresse}
% \changes{v4.8}{2007/06/25}{Unit\'e replaced by Centre}
% \changes{v4.10}{2007/12/14}{Moved addreses elsewhere}
%    \begin{macrocode}
% \ifx#21\fontsize{10pt}{12pt}\let\RR@optret\\\def\RR@Fr{(France)}%
% \else\fontsize{8pt}{10pt}\let\RR@optret\@comspace\let\RR@Fr\empty\fi
%  \selectfont
%  \csname CR@#1\endcsname
%   \ifx#21\\[2mm]\else \\\fi
%\fi}
%    \end{macrocode}
% \end{macro}\end{macro}
% \begin{macro}{\@@printCR}
% \changes{v4.9a}{2007/11/20}{macro added}
% Au premier janvier 2008, Futurs sera remplac\'e par trois centres de
% Recherche. L'une des deux macros suivantes est \`a utiliser avant, l'autre
% apr\'es cette date fatidique.
%    \begin{macrocode}
%\def\@@printCR{
% \@printCR 11\@printCR21\@printCR31\@printCR41\@printCR51\@printCR61
% \@printCR 10\@printCR20\@printCR30\@printCR40\@printCR50\@printCR60}
%\def\@@@printCR{
%   \@printCR11\@printCR21\@printCR31\@printCR41\@printCR51\@printCR61%
%   \@printCR71\@printCR81\@printCR10\@printCR20\@printCR30\@printCR40%
%   \@printCR50\@printCR60\@printCR70\@printCR80}
%    \end{macrocode}
% \end{macro}
% \subsection{Page de titre}
% \begin{macro}{\RR@issn}
% \changes{v4.3}{2006/05/03}{macro added}
% Ceci positionne le num\'ero ISSN et ISRN en bas \`a droite,
% en blanc (sur fond rouge).
%    \begin{macrocode}
\newcommand\RR@issn{%
 \setbox0\hbox{\rotatebox{90}{\issn@font\color{white}
    ISSN 0249-\RR@cond{6399}{0803}
    \qquad ISRN INRIA/\RR@cond{RR}{RT}-{}-\RR@No-{}-FR+ENG}}
    \dimen0=272mm \advance\dimen0 by -\ht0
 \RR@atxy{190mm}{\dimen0}{\box0}}
%    \end{macrocode}
% \end{macro}
% \begin{macro}{\@nothanks}
% \changes{v3.7e}{1999/05/07}{Modif de thanks}
% \changes{v3.7h}{2001/07/11}{Code of thanks changed}
% On red\'efinit la macro |\thanks| avec un argument optionnel. Ceci permet 
% d'inhiber cette macro. Note : |\thanks| est red\'efini plus loin \`a cause de
% |french|.
%    \begin{macrocode}
\def\@nothanks{\@ifnextchar[{\@xnothanks}{\@gobble}}
\def\@xnothanks[#1]#2{\relax}
\def\@xthanks[#1]#2{\orig@thanks{#2}%
  \expandafter\edef\csname @footnote@#1\endcsname{\the\c@footnote}}
\def\thanksref#1{\footnotemark[\csname @footnote@#1\endcsname]}
%    \end{macrocode}
% \end{macro}

% \begin{macro}{\ps@titrr}
% La commande |\ps@titrr| est utilis\'ee quand on fait
% |\pagestyle{titrr}|. Cette commande ne sert plus.
% \changes{v3.7b}{1998/02/11}{modif logo page 1}
% \changes{v5.0}{2011/09/28}{Not needed anymore}
% \end{macro}
% \begin{macro}{\@out@CR}
% La macro suivante avait pout but de positionner l'adresse du CR sur la page
% 1, et de calculer la taille restante sur la page. Dans la version actuelle
% on ne fait plus que calculer cette taille.
% \changes{v3.4}{1995/09/25}{Ajout de @out@CR}
% \changes{v4.3}{2006/05/03}{moved address upwards by 1cm}
% \changes{v5.0}{2011/09/28}{code simplified}
%    \begin{macrocode}
\newcommand\@out@CR{%
  \dimen0=25cm
  \advance\dimen0 -\ht0
  \dimen1=1in
  \advance\dimen1\topmargin
  \advance\dimen1\headheight
  \advance\dimen1\headsep
  \advance\dimen1\textheight
  \advance\dimen1\footskip
  \ifdim \dimen1>\dimen0 
     \advance\dimen1 -\dimen0 \enlargethispage{-\dimen1}\fi}
%    \end{macrocode}
% \end{macro}
% \begin{macro}{\makeRR}
% \begin{macro}{\makeRT}
% \begin{macro}{\rr@dash}
% \begin{macro}{\ifRR@RT}
% \begin{macro}{\RR@cond}
% Les commandes |\makeRR| et |\makeRT| cr\'eent la page de titre. On appelle une
% macro commune, en mettant un indicateur dans |\ifRR@RT|.
% \changes{v3.5}{1997/11/25}{Added new level of brace. This keeps defs local}
% \changes{v3.6}{1997/12/01}{changed brace position}
% \changes{v4.6}{2006/12/04}{command rr@dash added}
% \changes{v5.0}{2011/09/29}{command cpt@type replaced by ifRR@RT}
%    \begin{macrocode}
\newif\ifRR@RT
\newcommand\makeRR{\RR@RTtrue{\@makeRRorRT}}
\newcommand\makeRT{\RR@RTfalse{\@makeRRorRT}}
\newcommand\rr@dash{ --- }
\newcommand\RR@cond[2]{\ifRR@RT#1\else#2\fi}
%    \end{macrocode}
% \end{macro}
% \end{macro}
% \end{macro}
% \end{macro}
% \end{macro}
% \begin{macro}{\RR@start}
% Cette commande est appel\'ee au d\'ebut du document.
% D'abord, on se d\'ebrouille que les macros |\french| et |\english|
% existent.
% On d\'efinit |\and| comme s\'eparateur entre les noms des auteurs: une
% virgule qui supprime un \'eventuel espace qui pr\'ec\`ede.
% On passe en 1 colonne, si le mode est 2 colonnes ;
% on d\'esactive |\makeRR| et |\makeRT|.
% \changes{v3.7h}{2001/07/11}{language switch algo changed (durand)}
% \changes{v3.8a}{2002/11/12}{check double makeRR}
% \changes{v4.0}{2004/04/24}{eval@theme added}
% \changes{v4.11}{2009/06/09}{eval@theme removed}
% \changes{v3.7c}{1998/07/07}{Test de URmachin}
% \changes{v5.1}{2011/10/24}{Definition of french simplified}
% \changes{v5.1}{2012/01/17}{commnand and globally set (Fayolle)}
%    \begin{macrocode}
\def\RR@start{
   \global\let\makeRR\relax\global\let\makeRT\relax
   \let\orig@thanks\thanks
   \def\thanks{\@ifnextchar[ {\@xthanks}{\orig@thanks}}% ]
   \@ifundefined{english}{\def\english{}}{}
   \@ifundefined{french}{\def\french{}}{}
   \ifx\RR@rc@name\relax \@latex@error{Missing \string\URLorraine,
      \string\URRennes, \string \URRhoneAlpes, \string \URRocq, ^^J
      \string \URFuturs, or \string\URSophia}
      \@eha {\gdef\RR@rc@name{bordeaux}}%
   \fi
   \gdef\and{\unskip, }
   \@restonecolfalse
   \if@twocolumn\@restonecoltrue\onecolumn\fi}
%    \end{macrocode}
% \end{macro}
% \begin{macro}{\RR@fplow}
% Le cartouche de la premi\`ere page contient le num\'ero du rapport, la date
% et le nom de l'\'equipe-projet, en rouge sur fond blanc.
% \changes{v3.7c}{1998/04/28}{Utilisation de textdegree pour numero}
% \changes{v4.6}{2006/12/04}{French test added}
% \changes{v4.11}{2010/02/01}{Hack for empty date}
% \changes{v4.11c}{2010/04/19}{Spacing corrected}
%    \begin{macrocode}
\def\RR@fplow{%
  \RR@atxy{40mm}{240mm}{\hbox{\vbox{\parindent=0pt%
     \hsize=48mm \raggedright \hyphenpenalty 10000
     \color{red}%
 {\rrno@font \RR@cond{RESEARCH}{TECHNICAL}\\[0mm]REPORT \\[4mm]
      N\textdegree\ \RR@No} 
      \theme@font \\[2mm]
      \RR@date\\[3mm] \RR@prjtype~\RR@projet}}}}
%    \end{macrocode}
% \end{macro}
% \begin{macro}{\RR@fpmiddle}%
% Le reste de la premi\`ere page contient le titre du rapport et le nom des
% auteurs en blanc sur fond rouge.
% On inhibe |\thanks| et |\footnotemark| (pas de notes sur la page de
% couverture). On met le titre dans la bonne langue.
% On met le tout dans un bo\^ite de 14 cm de large et 10 de haut.
% \changes{v3.6}{1997/12/17}{changed def of footnotemark}
%    \begin{macrocode}
\def\RR@fpmiddle{%
 \RR@atxy{38mm}{113mm}{\hbox{\vbox to 10cm{\hsize=14cm  
   \begin{raggedright}
     \def\footnote##1{\relax}%
     \let\thanks\@nothanks
     \def\footnotemark{%
         \@ifnextchar[\my@xfootnotemark\relax} %]
     \def\my@xfootnotemark[##1]{\relax}%
     \color{white}
     {\title@font  \baselineskip43pt
        \ifRR@french {\RR@title}\else {\RR@etitle}\fi
       \\[1cm]}
     \baselineskip16pt
     {\author@font\RR@author}%
     \vfill
   \end{raggedright}}}}}
%    \end{macrocode}
% \end{macro}
% \begin{macro}{\RR@fsp}
% Allons-y pour la page 1. Toutes les modifications sont locales.
% On commence par red\'efinir les macros qui impriment
% les notes en bas de page, et les appels de note.
% \changes{v3.4}{1997/09/19}{Supression de fndot}
% \changes{v3.6}{1997/12/17}{reset footnotecounter}
%    \begin{macrocode}
\def\RR@sp{{%
\def\thefootnote{\fnsymbol{footnote}}
\setcounter{footnote}{0}
\def\@makefnmark{\hbox{\@textsuperscript{\@thefnmark}}}
\long\def\@makefntext##1{\parindent 1em\noindent
         \hb@xt@1.8em{%
             \hss\@textsuperscript{\normalfont\@thefnmark}} ##1}%
\let\footnoterule\relax 
%    \end{macrocode}
% La premi\`ere chose \`a faire est de calculer la taille de la page. 
% On positionne le logo avec l'adresse. On envoie ensuite le titre (langue
% principale)  et l'auteur.
% \changes{v5.1}{2011/10/21}{Informations shifted}
%    \begin{macrocode}
\null\@out@CR  \vskip 3cm \rr@frame
\begin{center}
\dimen0=\@themargin\advance \dimen0 by -26mm
\advance \rightskip by \dimen0
\advance \leftskip by -\dimen0\hsize 115mm
   {\Large\bf \ifRR@french \RR@title\else\RR@etitle\fi \par} 
   \vskip 2em
   {\large \lineskip .75em \RR@author \par}
   \vskip 1em
   \normalsize
%    \end{macrocode}
% On met le nom du projet. 
%    \begin{macrocode}
   \ifRR@french \RR@fprjtype\else\RR@prjtype\fi 
    ~\RR@projet \par
   \vskip 1em
%    \end{macrocode}
% On balance un truc du genre \og Rapport de recherche num\'ero 123 --- premier
% janvier 2000 --- 37 pages \fg
% \changes{v4.6}{2006/12/04}{French test added}
% \changes{v5.0}{2011/09/29}{French/English headers}
%    \begin{macrocode}
   \ifRR@french 
     Rapport \RR@cond{de recherche}{technique}
   \else 
     \RR@cond{Research}{Technical} Report
   \fi 
   \space n\textdegree{} \RR@No  \rr@dash
   \ifRR@version
        version \RR@version \rr@dash
     \ifRR@french 
        version initiale \RR@date\rr@dash version r\'evis\'ee \RR@dater
     \else
        initial version \RR@date \rr@dash revised version \RR@dater
     \fi
   \else
     \RR@date
   \fi
   \rr@dash \RR@nbpage\ pages\par
\end{center}
%    \end{macrocode}
% On envoie les notes.
%    \begin{macrocode}
\RR@note
\@thanks 
 \vfil
%    \end{macrocode}
%  On met le r\'esum\'e et les mots cl\'es dans la langue principale.
%  On utilise une largeur de texte tr\`es grande, car il reste souvent peu de
%  place sur la page.
% \changes{v3.4}{1997/09/19}{le moins de choses possible dans les test}
% \changes{v5.1}{2011/10/21}{Abstract shifted}
%    \begin{macrocode}
\dimen0=\@themargin
\advance \dimen0 by -11mm
\noindent\kern-\dimen0\parbox{150mm}{%
   \ifRR@french
      \RR@resume \par\vskip2mm \RR@motcle \else
      \RR@abstract\par\vskip2mm \RR@keyword\fi}}}
%    \end{macrocode}
% \end{macro}
% \begin{macro}{\@makeRRorRT}
% \changes{v3.7b}{1998/02/11}{Repositionnement logo page de titre}
% \changes{v5.0}{2011/09/11}{Plus de logo}
% \changes{v3.6}{1997/12/17}{Page de couverture change}
% \changes{v5.0}{2011/09/29}{Removed redef of makefnmark}
% \changes{v5.1}{2012/02/16}{Ajout (local) de topmargin}
% \changes{v4.3}{2006/05/03}{Logo file name changed}
% \changes{v2.5}{1997/10/27}{pagestyle-empty en trop}
% \changes{v2.5}{1997/10/27}{Suppression du tsvp, pto}
% Le code de cette macro est un peu long. Il a \'et\'e beaucoup simplifi\'e. On
% commence par la page de couverture.
%    \begin{macrocode}
\newcommand\@makeRRorRT{%
  \RR@start\topmargin=18pt
  \thispagestyle{empty}\pagestyle{empty}
%  \RR@atxy{0cm}{1.525cm}{\includegraphics{RR\RR@cond{}{t}firstpage}}
  \RR@atxy{0cm}{10mm}{\includegraphics{rrpage1}}
  \RR@atxy{38mm}{55mm}{\includegraphics[width=50mm]{logo-inria}}
  \RR@issn   \RR@fplow \RR@fpmiddle
  \null\vfill  \c@page\z@ \newpage
  \null\vfill  \c@page\z@ \newpage
  \RR@sp \vfil  \newpage
%    \end{macrocode}
%  On traite ici la page 2. On met le titre dans l'autre langue.
% Si le document est en fran\c cais, cette page est an anglais, sinon elle est
% dans la langue coutante, elle devrait \^etre en fran\c cais, mais on ne
% sais pas comment faire (autrefois, on modifiait |\language|, mais ce n'est
% pas bon). 
% \changes{v5.1}{2011/10/24}{Removed french sw}
%    \begin{macrocode}
   {\ifRR@french \english\fi
     {\Large\bf  \begin{center}
       \ifRR@french \RR@etitle\else \RR@title \fi
        \end{center}}
%    \end{macrocode}
%  Puis le r\'esum\'e et les mots cl\'es dans l'autre langue.
%    \begin{macrocode}
     \pagebreak[0]
     \ifRR@french \RR@abstract\par\vskip2mm \RR@keyword \french\else
     %   \french
\RR@resume \par\vskip2mm \RR@motcle\fi
   }
   \vfil \null \newpage
   \if@restonecol\twocolumn\fi
%    \end{macrocode}
%  Pour finir, on remet le compteur de notes \`a 0, on tue |\thanks|, et on
%  d\'esactive |\maketitle|.
% \changes{v3.5}{1997/11/25}{Global added}
%    \begin{macrocode}
  \setcounter{footnote}{0} 
  \global\def\thanks##1{\relax}}
%    \end{macrocode}
% \end{macro}
% \subsection{Les autre pages}
% \begin{macro}{\@inibe}
% Macro utilis\'ee pour mettre le titre sur toutes les pages. On vire les |\\|,
% les  |\thanks| et les |\footnotemark|.
%    \begin{macrocode}
\newcommand\@inibe{%
      \let\\\space
      \let\thanks\@nothanks%
      \def\footnotemark[##1]{\relax}}
%    \end{macrocode}
% \end{macro}
% \begin{macro}{\ps@pi}
% Style de page |pi|, i.e. pages int\'erieures. \`A droite, il y a le titre et
% le num\'ero de pages. \`A gauche il y a lun num\'ero de page et l'auteur.
% En bas, il y a \`a gauche le mot Inria, et \`a droite, il y a \og RR num\'ero 128 \fg,
% ou peut \^etre aussi \og RT \fg. Le num\'ero du rapport est remplac\'e par une cha\^\i ne
% magique. 
% \changes{v3.6}{1997/12/01}{Let ps@plain  ps@pi}
%    \begin{macrocode}
\newcommand\ps@pi{\let\@mkboth\@gobbletwo%
  \def\@oddhead{\vbox{\hbox to \textwidth{%
    \normalsize\normalfont{\itshape \@inibe\@titlehead}\hfil\thepage}%
    \hbox{\rule[-1ex]{\textwidth}{.03cm}}}}
  \def\@oddfoot{{\normalfont\footnotesize \RR@cond{RR}{RT}%
    \space n\textdegree{} \RR@No{}}\hfill}
  \def\@evenhead{\vbox{\hbox to \textwidth{%
    \normalsize\normalfont\thepage\hfil{\itshape \@inibe\@authorhead}}
    \hbox{\rule[-1ex]{\textwidth}{.03cm}}}}%
  \def\@evenfoot{\hfill{\footnotesize\normalfont Inria}}}
\let\ps@plain\ps@pi
%    \end{macrocode}
% \end{macro}
% \subsection{Le reste}
% Autre param\`etres.
%    \begin{macrocode}
\headheight1cm
\headsep1cm
\pagestyle{pi}
%    \end{macrocode}
% Autre param\`etres.
%    \begin{macrocode}
\@ifundefined{chapter}{\relax}
   {\def\chapter{\cleardoublepage \thispagestyle{pi}
   \global\@topnum\z@ \@afterindentfalse \secdef\@chapter\@schapter}}
%    \end{macrocode}
% Param\`etres de taille standard. Depuis la version 4.7 de mai 2007,
% ces valeurs ne sont plus utilis\'ees. Elles restent en commentaire
% pour pouvoir les copier dans les vieux documents.
% \changes{v4.7}{2007/05/11}{Sizes commented out}
%    \begin{macrocode}
%%\textwidth14cm
%%\textheight18cm
%%\evensidemargin0.96cm
%%\oddsidemargin0.96cm
%%\footskip2cm
%    \end{macrocode}
% Param\`etres de taille en mode RRA4.
% \changes{v3.7}{1998/01/19}{bigger footskip for RRA4}
% \changes{v4.7}{2007/05/11}{Sizes commented out}
%    \begin{macrocode}
%% Param\`etres de taille en mode RRA4.
%%\textwidth17cm
%%\textheight24cm
%%\evensidemargin-0.46cm
%%\oddsidemargin-0.46cm
%%\topmargin-1.46cm
%%\footskip1cm
%    \end{macrocode}
% Fraction et compteurs pour les flottants.
% Le style ne red\'efinit plus ces valeurs
% \changes{v5.0}{2011/09/28}{Comment\'e}
%    \begin{macrocode}
%%\setcounter{topnumber}{5}
%%\def\topfraction{1}
%%\setcounter{bottomnumber}{4}
%%\def\bottomfraction{1}
%%\setcounter{totalnumber}{10}
%%\def\textfraction{0}
%%\def\floatpagefraction{.5}
%%\clubpenalty=10000
%%\widowpenalty=10000
%%\hfuzz=1pt
%%\vfuzz=5pt
%    \end{macrocode}
%</RR>
% \Finale  
\endinput
